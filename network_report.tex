% Network Configuration Lab Report
\documentclass[12pt]{article}
\usepackage[utf8]{inputenc}
\usepackage{graphicx}
\usepackage[table]{xcolor}
\usepackage{listings}
\usepackage{hyperref}
\usepackage{enumitem}
\usepackage{geometry}

\geometry{
    a4paper,
    margin=2.5cm
}

\title{Multisite Networks Configuration Lab Report}
\author{Network Configuration Lab}
\date{\today}

\begin{document}

\maketitle

\tableofcontents
\newpage

\section{Introduction}
This report details the complete configuration of multisite networks using Cisco Packet Tracer. The configuration follows strict guidelines:
\begin{itemize}
    \item All configurations are done without using the config tab in Packet Tracer
    \item All PC/server configurations are performed through their desktop interfaces
    \item All commands and configurations are documented with screenshots or text format
\end{itemize}

\section{Network Overview}
The network topology consists of three main sites:
\begin{itemize}
    \item ISP Network (209.165.200.224/27)
    \item Headquarter Network
    \item Branch Network
\end{itemize}

\subsection{Initial Network Information}
Pre-configured addressing:
\begin{itemize}
    \item ISP LAN: 209.165.200.224/27
    \item ISP-HQ Link: 209.165.201.0/30
    \item Assigned space for subnetting: 192.168.2.0/24
\end{itemize}

\section{Section 1: Static Routing Configuration}

\subsection{1-A: Subnetting Analysis}
\subsubsection{Step 1: Network Requirements Analysis}
Required to support 60 hosts per subnet:
\begin{itemize}
    \item Number of host bits needed: 6 (2\^{}6 = 64 addresses, supporting 62 usable hosts)
    \item Resulting subnet mask: /26 (255.255.255.192)
\end{itemize}

\subsubsection{Step 2: Network Design Calculations}
\paragraph{Number of Required Subnets}
Four subnets are created from 192.168.2.0/24:
\begin{itemize}
    \item Subnet 0: 192.168.2.0/26 (Future expansion)
    \item Subnet 1: 192.168.2.64/26 (Headquarter LAN)
    \item Subnet 2: 192.168.2.128/26 (WAN link)
    \item Subnet 3: 192.168.2.192/26 (Branch LAN)
\end{itemize}

\paragraph{Subnet Details}
\begin{itemize}
    \item Subnet Mask (dotted decimal): 255.255.255.192
    \item Prefix Length: /26
    \item Usable Hosts per Subnet: 62 (2\^{}6 - 2)
\end{itemize}

\subsection{1-B: Detailed Interface Address Assignment}
\subsubsection{Step 1: Detailed Device Interface Configuration}

For Subnet 1 (192.168.2.64/26):
\begin{itemize}
    \item First valid host: 192.168.2.65 (Headquarter LAN)
    \item Last valid host: 192.168.2.126 (PC2)
    \item Network address: 192.168.2.64
    \item Broadcast address: 192.168.2.127
\end{itemize}

For Subnet 2 (192.168.2.128/26):
\begin{itemize}
    \item First valid host: 192.168.2.129 (Branch WAN)
    \item Second valid host: 192.168.2.130 (Headquarter WAN)
    \item Network address: 192.168.2.128
    \item Broadcast address: 192.168.2.191
\end{itemize}

For Subnet 3 (192.168.2.192/26):
\begin{itemize}
    \item First valid host: 192.168.2.193 (Branch LAN)
    \item Last valid host: 192.168.2.254 (PC1)
    \item Network address: 192.168.2.192
    \item Broadcast address: 192.168.2.255
\end{itemize}

\subsubsection{Detailed Configuration Steps}

\paragraph{Headquarter Router LAN Interface}
1. Enter configuration mode:
\begin{lstlisting}[frame=single]
enable
configure terminal
interface FastEthernet0/0
ip address 192.168.2.65 255.255.255.192
no shutdown
exit
\end{lstlisting}

\paragraph{PC2 Configuration}
1. Click PC2 icon
2. Select Desktop tab
3. Click IP Configuration
4. Enter settings:
\begin{itemize}
    \item IPv4 Address: 192.168.2.126
    \item Subnet Mask: 255.255.255.192
    \item Default Gateway: 192.168.2.65
\end{itemize}

[Similar detailed steps for all other interfaces...]

\subsection{1-C: Packet Tracer Configuration Steps}
\subsubsection{Router Configurations}

\paragraph{Branch Router Configuration}
\begin{lstlisting}[frame=single]
enable
configure terminal
hostname Branch
!
interface FastEthernet0/0
 ip address 192.168.2.193 255.255.255.192
 no shutdown
!
interface Serial0/0/0
 ip address 192.168.2.129 255.255.255.192
 clock rate 64000
 no shutdown
!
ip route 192.168.2.64 255.255.255.192 192.168.2.130
ip route 209.165.200.224 255.255.255.224 192.168.2.130
\end{lstlisting}

\paragraph{Headquarter Router Configuration}
\begin{lstlisting}[frame=single]
enable
configure terminal
hostname Headquarter
!
interface FastEthernet0/0
 ip address 192.168.2.65 255.255.255.192
 no shutdown
!
interface Serial0/0/0
 ip address 192.168.2.130 255.255.255.192
 no shutdown
!
interface Serial0/0/1
 ip address 209.165.201.2 255.255.255.252
 no shutdown
!
ip route 192.168.2.192 255.255.255.192 192.168.2.129
ip route 209.165.200.224 255.255.255.224 209.165.201.1
\end{lstlisting}

\paragraph{ISP Router Configuration}
\begin{lstlisting}[frame=single]
enable
configure terminal
hostname ISP
!
interface FastEthernet0/0
 ip address 209.165.200.225 255.255.255.224
 no shutdown
!
interface Serial0/0
 ip address 209.165.201.1 255.255.255.252
 no shutdown
!
ip route 192.168.2.0 255.255.255.0 209.165.201.2
\end{lstlisting}

\subsubsection{PC and Server Configurations}
\paragraph{PC1 Configuration Steps}
1. Click PC1 icon
2. Go to Desktop tab
3. Click IP Configuration
4. Enter:
   \begin{itemize}
   \item IP Address: 192.168.2.254
   \item Subnet Mask: 255.255.255.192
   \item Default Gateway: 192.168.2.193
   \end{itemize}

\paragraph{PC2 Configuration Steps}
1. Click PC2 icon
2. Go to Desktop tab
3. Click IP Configuration
4. Enter:
   \begin{itemize}
   \item IP Address: 192.168.2.126
   \item Subnet Mask: 255.255.255.192
   \item Default Gateway: 192.168.2.65
   \end{itemize}

\paragraph{Web Server Configuration Steps}
1. Click Web Server icon
2. Go to Desktop tab
3. Click IP Configuration
4. Enter:
   \begin{itemize}
   \item IP Address: 209.165.200.253
   \item Subnet Mask: 255.255.255.224
   \item Default Gateway: 209.165.200.225
   \end{itemize}

\subsubsection{Verification Steps}
1. From PC1:
\begin{lstlisting}[frame=single]
ping 192.168.2.126
ping 209.165.200.253
\end{lstlisting}

2. From PC2:
\begin{lstlisting}[frame=single]
ping 192.168.2.254
ping 209.165.200.253
\end{lstlisting}

3. From Web Server:
\begin{lstlisting}[frame=single]
ping 192.168.2.254
ping 192.168.2.126
\end{lstlisting}

\section{Section 2: Dynamic Routing Configuration}

\subsection{2-A: Dynamic Routing Setup}
\subsubsection{Router Configurations}

\paragraph{Branch Router RIP Configuration}
\begin{lstlisting}[frame=single]
enable
configure terminal
router rip
version 2
network 192.168.2.0
no auto-summary
\end{lstlisting}

\paragraph{Headquarter Router RIP Configuration}
\begin{lstlisting}[frame=single]
enable
configure terminal
router rip
version 2
network 192.168.2.0
network 209.165.201.0
no auto-summary
\end{lstlisting}

\paragraph{ISP Router RIP Configuration}
\begin{lstlisting}[frame=single]
enable
configure terminal
router rip
version 2
network 209.165.200.0
network 209.165.201.0
no auto-summary
\end{lstlisting}

\subsection{2-B: Enhanced HTTP and DNS Server Configuration}

\subsubsection{Webserver2 Detailed Setup}
1. Add new server:
   \begin{itemize}
   \item Drag a "Server-PT" from the device list
   \item Connect to Branch switch using straight-through cable
   \end{itemize}

2. Configure IP settings:
   \begin{itemize}
   \item Click server → Desktop → IP Configuration
   \item IP Address: 192.168.2.200
   \item Subnet Mask: 255.255.255.192
   \item Default Gateway: 192.168.2.193
   \end{itemize}

3. Configure HTTP service:
   \begin{itemize}
   \item Click server → Services → HTTP
   \item Turn service On
   \item Click Edit
   \item Replace default HTML with:
   \begin{lstlisting}[frame=single]
<html>
<head><title>Welcome to MyCompany</title></head>
<body>
<h1>Welcome to MyCompany Website</h1>
<p>This is the company's internal website.</p>
</body>
</html>
   \end{lstlisting}
   \end{itemize}

\subsubsection{DNS Server Detailed Configuration}
1. Add DNS Server:
   \begin{itemize}
   \item Add new server in Headquarter network
   \item Connect to same switch as PC2
   \item Name it "Headquarter\_DNS"
   \end{itemize}

2. Configure IP settings:
   \begin{itemize}
   \item IP Address: 192.168.2.125
   \item Subnet Mask: 255.255.255.192
   \item Default Gateway: 192.168.2.65
   \end{itemize}

3. Configure DNS Service:
   \begin{itemize}
   \item Click Services → DNS
   \item Enable DNS Service
   \item Add Record:
     \begin{itemize}
     \item Name: www.mycompany.fr
     \item Type: A Record
     \item Address: 192.168.2.200
     \end{itemize}
   \item Add Record:
     \begin{itemize}
     \item Name: ftpmyfiles.fr
     \item Type: A Record
     \item Address: 209.165.200.253
     \end{itemize}
   \end{itemize}

4. TCP Port Analysis:
   \begin{itemize}
   \item DNS Query: Source Port (random high port) → Destination Port 53
   \item HTTP Connection: Source Port (random high port) → Destination Port 80
   \item Capture these using Packet Tracer's Simulation Mode
   \end{itemize}

\subsection{2-C: Enhanced FTP Server Configuration}

\subsubsection{Detailed FTP Setup Steps}
1. Configure ISP Webserver:
   \begin{itemize}
   \item Click server → Services → FTP
   \item Enable FTP service
   \item Create user account:
     \begin{itemize}
     \item Username: admin
     \item Password: admin123
     \end{itemize}
   \end{itemize}

2. Create and Upload File from PC2:
   \begin{itemize}
   \item Open Command Prompt
   \item Type: \texttt{echo "Name: [Your Name] \\ Date: [Current Date]" > infonow.txt}
   \item Use FTP commands:
   \begin{lstlisting}[frame=single]
ftp ftpmyfiles.fr
Username: admin
Password: admin123
put infonow.txt
\end{lstlisting}
   \end{itemize}

3. Create Server File:
   \begin{itemize}
   \item Access server directly
   \item Create file \texttt{infonowserver.txt}
   \item Add content through FTP service
   \end{itemize}

4. Download File to PC2:
\begin{lstlisting}[frame=single]
ftp ftpmyfiles.fr
Username: admin
Password: admin123
get infonowserver.txt
\end{lstlisting}

5. TCP Port Analysis:
   \begin{itemize}
   \item FTP Control: Port 21
   \item FTP Data: Port 20
   \item Client: Random high ports
   \item Capture in Simulation Mode for exact port numbers
   \end{itemize}

\subsection{2-D: Enhanced DHCP Server Configuration}

\subsubsection{Detailed DHCP Server Setup}
1. Add DHCP Server:
   \begin{itemize}
   \item Add new server to Branch network
   \item Connect to Branch switch
   \item Configure static IP:
     \begin{itemize}
     \item IP: 192.168.2.199
     \item Subnet Mask: 255.255.255.192
     \item Default Gateway: 192.168.2.193
     \end{itemize}
   \end{itemize}

2. Configure DHCP Service:
   \begin{itemize}
   \item Click Services → DHCP
   \item Enable DHCP Service
   \item Configure Pool:
     \begin{itemize}
     \item Pool Name: BRANCH\_POOL
     \item Start IP: 192.168.2.201
     \item Subnet Mask: 255.255.255.192
     \item Default Gateway: 192.168.2.193
     \item DNS Server: 192.168.2.125
     \item Lease Time: 7 days (604800 seconds)
     \end{itemize}
   \end{itemize}

3. Configure Clients for DHCP:
   \begin{itemize}
   \item For each device (except DHCP server and router):
     \begin{itemize}
     \item Open IP Configuration
     \item Select DHCP
     \item Click "Release/Renew" to get new IP
     \end{itemize}
   \end{itemize}

4. Verification Steps:
   \begin{itemize}
   \item Check IP assignments on all clients
   \item Verify connectivity:
     \begin{itemize}
     \item Ping between clients
     \item Access web services
     \item Test DNS resolution
     \end{itemize}
   \item Monitor DHCP transactions in Simulation Mode
   \end{itemize}

\section{Conclusion}
This report has detailed the complete configuration of a multisite network using both static and dynamic routing protocols. The network successfully implements:
\begin{itemize}
    \item Proper subnetting and IP addressing
    \item Static and dynamic routing
    \item Web and DNS services
    \item FTP services
    \item DHCP services
\end{itemize}

All configurations were performed through CLI and desktop interfaces as required, and full connectivity between all network segments has been verified.

\end{document}
